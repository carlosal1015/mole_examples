\documentclass[a4paper, 10pt,
    twoside,
    logo={logouni},
    logo height=2cm,
    title in boldface,
    title in sffamily,
    theorem in new line,
    %   remove qed,
    %   remove problem qed,
    %   remove solution qed
    colored solution={DarkBlue},
    %   hide solution,
]{homework}

\usepackage{unicode-math}
\usepackage{diffcoeff}
\usepackage[svgnames]{xcolor}
\usepackage{minted}
\usepackage[linesnumbered,ruled,boxed,vlined,spanish]{algorithm2e}
\usepackage{algorithmicx}
\usepackage[
    citestyle=numeric,
    style=numeric,
    backend=biber,
]{biblatex}
\usepackage{booktabs}
\usepackage{hyperref}

\setminted{breaklines=true}
\setminted[octave]{highlightcolor=yellow!50!white}
\setminted[cpp]{highlightcolor=yellow!50!white}
\setmonofont{Fira Code}[Contextuals=Alternate,Scale=MatchLowercase]
\renewcommand{\listingscaption}{Programa}
\difdef{c}{L}{op-symbol=\mathop{}\!\mathbin\bigtriangleup}

\ExplSyntaxOn
\NewDocumentCommand{\mintedpath}{m}
{
	\seq_gset_split:Nnn \g_paulie_mintedpath_seq { } { #1 }
	\seq_gput_left:Nn \g_paulie_mintedpath_seq { }
}

\seq_new:N \g_paulie_mintedpath_seq

\NewDocumentCommand{\pathinputminted}{O{}mm}
{
	\seq_map_inline:Nn \g_paulie_mintedpath_seq
	{
		\file_if_exist:nT { ##1 #3 }
		{
			\inputminted[#1]{#2}{##1 #3}
			\seq_map_break:
		}
	}
}
\ExplSyntaxOff

\mintedpath{{../examples/cpp/}{../examples/octave/}}

\graphicspath{{../examples/octave/}}
\addbibresource{../tutorial/references.bib}

\hypersetup{
    pdfencoding=auto,
    linktocpage=true,
    colorlinks=true,
    linkcolor=DarkBlue,
    urlcolor=magenta,
    pdfpagelabels,
    pdftex,
    pdfauthor={Carlos Aznarán Laos},
    pdftitle={Métodos Miméticos para resolver EDPs y la librería MOLE},
    pdfsubject={Lecture},
    pdfkeywords={ode, pde},
    pdfproducer={LuaHBTeX, Version 1.18.0 (TeX Live 2024/Arch Linux)},
    % bookmark=false
}

\UseLanguage{Spanish}
\title{Métodos Miméticos para resolver EDPs y la librería MOLE, \href{https://carlosal1015.github.io/mole_examples/main.pdf}{Semana $1$}}
\author{Carlos Aznarán}
\date{Lima, \TheDate{2025-01-19}}
