\section{Instalación}

La única dependencia dura es
\href{https://arma.sourceforge.net/docs.html}{\mintinline{cpp}|armadillo|}.
Los pasos están descritos en el Programa~\ref{code:installer.sh}.

\begin{listing}[ht!]
    \tiny
    \centering
    \pathinputminted[frame=single,framesep=10pt,linenos,firstline=1,lastline=18,highlightlines={9,15}]{bash}{installer.sh}
    \caption{Instalación de MOLE vía \texttt{installer.sh} en \href{https://archlinux.org}{Arch Linux}.}
    \label{code:installer.sh}
\end{listing}

\section{Programas utilizados}

\subsection{Programa~\ref{code:elliptic1D.m}}

\begin{listing}[ht!]
    \tiny
    \centering
    \pathinputminted[frame=single,framesep=10pt,linenos,firstline=1,lastline=53,highlightlines={21,29}]{octave}{../examples/octave/elliptic1D.m}
    \caption{Programa~\texttt{elliptic1D.m}}
    \label{code:elliptic1D.m}
\end{listing}

\begin{listing}[ht!]
    \tiny
    \centering
    \pathinputminted[frame=single,framesep=10pt,linenos,firstline=1,lastline=80,highlightlines={24,27}]{cpp}{../examples/cpp/elliptic1D.cpp}
    \caption{Programa~\texttt{elliptic1D.cpp}}
    \label{code:elliptic1D.cpp}
\end{listing}

\begin{listing}[ht!]
    \tiny
    \centering
    \pathinputminted[firstline=1,lastline=8,highlightlines={7}]{cmake}{../examples/cpp/CMakeLists.txt}
    \pathinputminted[firstline=14,lastline=14]{cmake}{../examples/cpp/CMakeLists.txt}
    \pathinputminted[firstline=16,lastline=16,highlightlines={16}]{cmake}{../examples/cpp/CMakeLists.txt}
    \pathinputminted[firstline=65,lastline=69]{cmake}{../examples/cpp/CMakeLists.txt}
    \caption{Programa~\texttt{CMakeLists.txt}}
    \label{code:CMakeLists.txt}
\end{listing}

\begin{listing}[ht!]
    \tiny
    \centering
    \pathinputminted[frame=single,framesep=10pt,linenos]{python}{elliptic1dplots.py}
    \caption{Programa~\texttt{elliptic1dplots.py}}
    \label{code:elliptic1dplots.py}
\end{listing}

\begin{listing}[ht!]
    \tiny
    \centering
    \pathinputminted{text}{octave-help.txt}
    \caption{\href{https://raw.githubusercontent.com/carlosal1015/mole_examples/refs/heads/main/homework/octave-help.txt}{\mintinline{text}|octave-help.txt|}
        muestra la lista de opciones de Octave por la línea de comandos.}
    \label{code:octave-help.txt}
\end{listing}

\begin{listing}[ht!]
    \tiny
    \centering
    \pathinputminted{text}{moledirectoriesoctave.txt}
    \caption{\href{https://raw.githubusercontent.com/carlosal1015/mole_examples/refs/heads/main/homework/moledirectoriesoctave.txt}{\mintinline{text}|moledirectoriesoctave.txt|}
        muestran la estructura de árbol de directorios de las funciones Octave / MATLAB de la biblioteca MOLE.}
    \label{code:moledirectoriesoctave.txt}
\end{listing}

\begin{listing}[ht!]
    \tiny
    \centering
    \pathinputminted{text}{moledirectoriescpp.txt}
    \caption{\href{https://raw.githubusercontent.com/carlosal1015/mole_examples/refs/heads/main/homework/moledirectoriescpp.txt}{\mintinline{text}|moledirectoriescpp.txt|}
        muestran la estructura de árbol de directorios de las cabeceras de C++ de la biblioteca MOLE.}
    \label{code:moledirectoriescpp.txt}
\end{listing}

\begin{listing}[ht!]
    \tiny
    \centering
    \pathinputminted[frame=single,framesep=10pt,linenos,firstline=1,lastline=53,highlightlines={21,29}]{octave}{elliptic1DAznaran.m}
    \caption{Programa~\texttt{elliptic1DAznaran.m}}
    \label{code:elliptic1DAznaran.m}
\end{listing}
