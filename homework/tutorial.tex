% arara: clean: {
% arara: --> extensions:
% arara: --> ['aux', 'bbl', 'bcf', 'blg','idx', 'log', 'mst', 'pdf', 'run.xml']
% arara: --> }
% arara: lualatex: {
% arara: --> shell: yes,
% arara: --> draft: yes,
% arara: --> interaction: batchmode
% arara: --> }
% arara: biber
% arara: lualatex: {
% arara: --> shell: yes,
% arara: --> draft: no,
% arara: --> interaction: batchmode
% arara: --> }
% arara: lualatex: {
% arara: --> shell: yes,
% arara: --> draft: no,
% arara: --> interaction: batchmode
% arara: --> }
% arara: clean: {
% arara: --> extensions:
% arara: --> ['aux', 'bbl', 'bcf', 'blg','idx', 'log', 'mst', 'run.xml']
% arara: --> }
\documentclass[a4paper,abstract=true]{scrreprt}
\usepackage{fontspec}
\usepackage[svgnames]{xcolor}
\usepackage{graphicx}
\usepackage{mathtools}
\usepackage[ISO]{diffcoeff}
\usepackage{minted}
\usepackage[
    citestyle=numeric,
    style=numeric,
    backend=biber,
]{biblatex}
\usepackage{hyperref}

\title{\color{DarkBlue}MOLE Tutorial}
\titlehead{\centering
\includegraphics[width=.3\paperwidth]{logo.png}}
\author{based on commit \href{https://github.com/csrc-sdsu/mole/tree/b1176d9969e807fb62bed8fee28bc0eb9648a93a}{\mintinline{bash}|b1176d9|}}

\setminted{breaklines=true}
\setminted[octave]{highlightcolor=yellow!50!white}
\setminted[cpp]{highlightcolor=yellow!50!white}
\setmonofont{Fira Code}[Contextuals=Alternate,Scale=MatchLowercase]
\addbibresource{references.bib}

\begin{document}
\maketitle

\begin{abstract}
	This document is written for newcomers in MOLE stands
	Mimetic Operators Library Enhanced, but with a solid foundation
	in numerical analysis for PDEs.
	The algorithms are continuously grows in size and quality and the
	examples are diverse implemented independently in
	\href{https://octave.org}{GNU Octave}/\href{https://www.mathworks.com/products/matlab.html}{MATLAB}
	and \href{https://isocpp.org}{C++} through Armadillo sparse
	linear algebra library.
	For now, these are exclusively the two language flavours.
	A clear comprenhension will gave you the capabilities to the translation
	into other high-performance scientific languages such as \href{https://julialang.org}{Julia},
	\href{https://www.python.org}{Python}, \href{https://fortran-lang.org}{Fortran}, \href{https://www.open-std.org/jtc1/sc22/wg14}{C}
	and \href{https://www.rust-lang.org}{Rust}.
\end{abstract}

\part{First part}

\chapter{Foo}

\section{Meeting MOLE}

The official website is \url{https://csrc-sdsu.github.io/mole}.
After skimming the description and reading the papers you will find out that this method never uses a ghost points.

\subsection{Create the staggered grid}


\part{Numerical Exercises}

\chapter{Foo}

\section{Transport 1D}

\begin{equation*}
	\diffp{u}{t}+c\diffp{u}{x}=
	0.
\end{equation*}

\begin{listing}[ht!]
	\tiny
	\centering
	\inputminted[frame=single,framesep=10pt,linenos,firstline=9,lastline=35,highlightlines={9}]{cpp}{../examples/cpp/hyperbolic1D_upwind.cpp}
	\caption{Programa~\texttt{hyperbolic1Dupwind.cpp}}
	\label{code:hyperbolic1Dupwind.cpp}
\end{listing}

\section{Poisson 1D}

\begin{listing}[ht!]
	\tiny
	\centering
	\inputminted[frame=single,framesep=10pt,linenos,firstline=1,lastline=53,highlightlines={21,29}]{octave}{../examples/octave/elliptic1D.m}
	\caption{Programa~\texttt{elliptic1D.m}}
	\label{code:elliptic1D.m}
\end{listing}

\section{The 1D Diffusion Equation}

Given the one-dimensional heat equation

\begin{equation}\label{eq:IBVPheat1d}
	\begin{cases}
		\diffp{u}{t}=
		\kappa\diffp[2]{u}{x},
		                              & \left(x,t\right)\in
		\left(a,b \right)\times\left(0, \infty \right).     \\
		u\left(x,0\right)=f(x),
		                              & x\in
		\left[a,b\right].                                   \\
		u\left(a,t\right)= \alpha(t), & t\in
		\left(0,\infty\right),                              \\
		u\left(b,t\right)= \beta(t),
		                              & t\in
		\left(0,\infty\right),
	\end{cases}
\end{equation}

where $\kappa$ is the thermal diffusivity.\\

The 1D heat equation code by mimetic methods


\begin{listing}[ht!]
	\tiny
	\centering
	\inputminted[frame=single,framesep=12pt,linenos,firstline=1,lastline=80,highlightlines={20,29}]{octave}{../examples/octave/parabolic1D.m}
	\caption{Programa~\texttt{parabolic1D.m}}
	\label{code:parabolic1D.m}
\end{listing}


In the one-dimensional heat equation in the parabolic1D.m code, the PDE being solved is given by:


\begin{equation}\label{eq:IBVPheat1d}
	\begin{cases}
		\diffp{u}{t}=
		\kappa\diffp[2]{u}{x},
		                         & \left(x,t\right)\in
		\left(0,1 \right)\times\left(0, 1 \right).     \\
		u\left(x,0\right)= 0,
		                         & x\in
		\left[0,1\right].                              \\
		u\left(0,t\right)= 100., & t\in
		\left(0,1 \right),                             \\
		u\left(1,t\right)= 100.,
		                         & t\in
		\left(0, 1\right),
	\end{cases}
\end{equation}


\textbf{Line 11:} In this part the code defines the value of the diffusion coefficient $\kappa =1$.\\

\textbf{Line 12:} Define the left side $a= 0$ of the domain of the variable $x$.\\

\textbf{Line 13:} Define the right hand side $b= 1$ of the domain of the variable $x$.\\

\textbf{Line 15:} The Operator's order of accuracy $k = 2$.\\

\textbf{Line 16:} $m$ is the number of cells, where $m$ can take values ​​$m \geq 2*k+1$; in this case $m=2*(2)+1=5$.\\

\textbf{Line 17:} In this line the step size in $x$ is defined, defined as $dx=(b-a)/m$, en this case, replacing it, we have $dx =\frac{1-0}{5} =\frac{1}{5}$ \\

\textbf{Line 19:} End time $t=1$.\\

\textbf{Line 20:} Von Neumann stability criterion for $k=2$, we have $ dt = \frac{(dx)^{2}}{3 \kappa}$, in this exercise the step in time is given by: $dt =\frac{1}{75}$\\

\textbf{Line 22:} $L = lap(k,m,dx)$ 1D mimetic Laplacian operator is of order $m+2$ so $m+2$ for this exercise is of order 7 by 7, where $k=2$, $m = 5$ and $\frac{1}{5}$, then $L = lap(2,5,\frac{1}{5})$.\\

\textbf{Line 25:} The initial condition  value $u(x,0)=0$ is a matrix of order $m+2$ by $1$, in this case $7$ by $1$.\\

\textbf{Line 27:} The boundary condition is on the left side u(a,t)= u(0,t)=100.\\

\textbf{Line 28:} The boundary condition on the right side u(b,t)= u(1,t)=100.\\

\textbf{Line 31:} The mesh used in the mimetic method  grid = $[west$  $west+dx/2: dx :east-dx/2$  $east]$, then grid = $[a$ $a+\frac{dx}{2}: dx : b-\frac{dx}{2}$ $b  ]$, in this exercise the mesh is given by:

\begin{center}

	grid = $[0$ $\frac{1}{10}: \frac{1}{5}: \frac{9}{10}$ $1]$ = $  \{0, \frac{1}{10}, \frac{3}{10}, \frac{5}{10}, \frac{7}{10}, \frac{9}{10}, 1 \}$

\end{center}

\textbf{Line 34:} If explicit=1 then the PDE will be solved in time by the explicit method, and if explicit=0 then the PDE will be solved by the implicit method.\\

\textbf{Line 37 to 53:} In this section we have the solution of the PDE and the graph. \\

\textbf{Line 40:} The value of L is obtained by:

\begin{equation}
	\frac{\partial u}{\partial t} = \kappa \frac{\partial^{2} u}{\partial x^{2}}
\end{equation}

Discretizing $\frac{\partial u}{\partial t}$ by forward finite differences, and $\frac{\partial^{2} u}{\partial x^{2}}$ by applying mimetic methods

\begin{equation}
	\frac{u^{n+1}_{i} - u^{n}_{i} }{dt} = \kappa L 	u^{n}_{i}
\end{equation}

then clearing $u^{n+1}_{i}$ we obtain:


\begin{equation}
	u^{n+1}_{i} = u^{n}_{i}  +  \kappa dt L u^{n}_{i}
\end{equation}

Factoring $u^{n}_{i}$ we have:

\begin{equation}
	u^{n+1}_{i} = (I  +  \kappa dt L )u^{n}_{i}
\end{equation}

where $I$ is the identity matrix of general order $m+2$ by $m+2$, for this exercise of order $7$ by $7$, being

$$  L = (\kappa dt L +I )$$

being in the code: $ L = alpha*dt*L + speye(size(L))$; finally to obtain the solution

\begin{equation}
	u^{n+1}_{i} = L *u^{n}_{i}
\end{equation}

in the code on line 51 we have $ U = L*U$






\section{Convection-diffusion}

\chapter{Tests}

\chapter{Python}

\chapter{Fortran}

\chapter{C++}

\chapter{Octave}

\nocite{*}
\printbibliography[title={Referencias}]

\end{document}
