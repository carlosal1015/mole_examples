% arara: clean: {
% arara: --> extensions:
% arara: --> ['aux', 'bbl', 'bcf', 'blg','idx', 'log', 'mst', 'pdf', 'run.xml']
% arara: --> }
% arara: lualatex: {
% arara: --> shell: yes,
% arara: --> draft: yes,
% arara: --> interaction: batchmode
% arara: --> }
% arara: biber
% arara: lualatex: {
% arara: --> shell: yes,
% arara: --> draft: no,
% arara: --> interaction: batchmode
% arara: --> }
% arara: lualatex: {
% arara: --> shell: yes,
% arara: --> draft: no,
% arara: --> interaction: batchmode
% arara: --> }
% arara: clean: {
% arara: --> extensions:
% arara: --> ['aux', 'bbl', 'bcf', 'blg','idx', 'log', 'mst', 'run.xml']
% arara: --> }
\documentclass[a4paper,abstract=true]{scrreprt}
\usepackage{fontspec}
\usepackage[svgnames]{xcolor}
\usepackage{graphicx}
\usepackage{mathtools}
\usepackage[ISO]{diffcoeff}
\usepackage{minted}
\usepackage[
    citestyle=numeric,
    style=numeric,
    backend=biber,
]{biblatex}
\usepackage{hyperref}

\title{\color{DarkBlue}MOLE Tutorial}
\titlehead{\centering    
\includegraphics[width=.3\paperwidth]{logo.png}}
\author{based on commit \href{https://github.com/csrc-sdsu/mole/tree/b1176d9969e807fb62bed8fee28bc0eb9648a93a}{\mintinline{bash}|b1176d9|}}

\setminted{breaklines=true}
\setminted[octave]{highlightcolor=yellow!50!white}
\setminted[cpp]{highlightcolor=yellow!50!white}
\setmonofont{Fira Code}[Contextuals=Alternate,Scale=MatchLowercase]
\addbibresource{references.bib}

\begin{document}
\maketitle

\begin{abstract}
    This document is written for newcomers in MOLE stands
    Mimetic Operators Library Enhanced, but with a solid foundation
    in numerical analysis for PDEs.
    The algorithms are continuously grows in size and quality and the
    examples are diverse implemented independently in
    \href{https://octave.org}{GNU Octave}/\href{https://www.mathworks.com/products/matlab.html}{MATLAB}
    and \href{https://isocpp.org}{C++} through Armadillo sparse
    linear algebra library.
    For now, these are exclusively the two language flavours.
    A clear comprenhension will gave you the capabilities to the translation
    into other high-performance scientific languages such as \href{https://julialang.org}{Julia},
    \href{https://www.python.org}{Python}, \href{https://fortran-lang.org}{Fortran}, \href{https://www.open-std.org/jtc1/sc22/wg14}{C}
    and \href{https://www.rust-lang.org}{Rust}.
\end{abstract}

\part{First part}

\chapter{Foo}

\section{Meeting MOLE}

The official website is \url{https://csrc-sdsu.github.io/mole}.
After skimming the description and reading the papers you will find out that this method never uses a ghost points.

\subsection{Create the staggered grid}


\part{Numerical Exercises}

\chapter{Foo}

\section{Ecuación de Poisson 1D}

\begin{listing}[ht!]
    \tiny
    \centering
    \inputminted[frame=single,framesep=10pt,linenos,firstline=1,lastline=53,highlightlines={21,29}]{octave}{../examples/octave/elliptic1D.m}
    \caption{Programa~\texttt{elliptic1D.m}}
    \label{code:elliptic1D.m}
\end{listing}

\section{The 1D Diffusion Equation}

Given the one-dimensional heat equation
\begin{equation}\label{eq:IBVPheat1d}
    \begin{cases}
        \diffp{u}{t}=
        \kappa\diffp[2]{u}{x},
         & \left(x,t\right)\in
        \left(a,b \right)\times\left(0, \infty \right). \\
        u\left(x,0\right)=f(x),
         & x\in
        \left[a,b\right].                       \\
        u\left(a,t\right)= \alpha(t), & t\in
        \left(0,\infty\right),\\
        u\left(b,t\right)= \beta(t),
         & t\in
        \left(0,\infty\right),
    \end{cases}
\end{equation}

where $\kappa=1$ is the thermal diffusivity.

The 1D heat equation code by mimetic methods

\begin{listing}[ht!]
    \tiny
    \centering
    \inputminted[frame=single,framesep=10pt,linenos,firstline=1,lastline=80,highlightlines={20,29}]{octave}{../examples/octave/parabolic1D.m}
    \caption{Programa~\texttt{parabolic1D.m}}
    \label{code:parabolic1D.m}
\end{listing}


En  la ecuación de calor unidimensional en el código parabolic1D.m

\begin{equation}
	\frac{\partial u}{\partial t} = \alpha \frac{\partial^{2} u}{\partial x^{2}},   x \in [0,1] ; t \in [0,1]
\end{equation}

con condición inicial: $u(x,0) = 0$\\

con condición de contorno: $u(0,t)= 100$ y $u(1,t) =100$\\

donde 

$k = 2$: es el orden de precisión

 $m = 2*k+1=2*(2)+1 = 5$: Número de celdas
 
$dx =\frac{1-0}{5} =\frac{1}{5}$ : Tamaño del paso 

$dt = \frac{(dx)^{2}}{3*\alpha} =\frac{1}{75}$\\

\textbf{\textbf{Operador Laplaciano}}\\

$L = lap(k,m,dx)$  Operador laplaciano mimética 1D es de orden $m+2$ por $m+2$ para este ejemplo es de orden  7 por 7,  se tiene $L = D* G$, donde $D = div(k,m,dx)$ es de orden en general $m+2$ por $m+1$, en particular 7 por 6, y $G = grad(k,m,dx)$ es de orden $m+1$ por $m+2$\\


grafica de la malla:\\


 grid = $[west$  $west+dx/2: dx :east-dx/2$  $east]$
 
entonces malla = $[0$ $0+\frac{1}{10}: \frac{1}{5}: 1-\frac{1}{10}$ $1  ]$ = $[0$ $\frac{1}{10}: \frac{1}{5}: \frac{9}{10}$ $1]$ = $  \{0, \frac{1}{10}, \frac{3}{10}, \frac{5}{10}, \frac{7}{10}, \frac{9}{10}, 1 \}$


Método explicito en la ecuación unidimensional

\begin{equation}
	\frac{\partial u}{\partial t} = \alpha \frac{\partial^{2} u}{\partial x^{2}}
\end{equation}
Discretizando  $\frac{\partial u}{\partial t}$ por diferencias finitas hacia adelante, y $\frac{\partial^{2} u}{\partial x^{2}}$ aplicando método miméticos

\begin{equation}	
	\frac{u^{n+1}_{i} - u^{n}_{i} }{dt} = \alpha L 	u^{n}_{i}
\end{equation}

 luego despejando $u^{n+1}_{i}$ se obtiene:
 
 \begin{equation}	
 	u^{n+1}_{i} = u^{n}_{i}  +  \alpha dt L u^{n}_{i}
 \end{equation}

factorizando $u^{n}_{i}$ tenemos:

 \begin{equation}	
	u^{n+1}_{i} = (I  +  \alpha dt L )u^{n}_{i}
\end{equation}

donde $I$ es la matriz identidad de orden en general $m+2$ por $m+2$, para este ejemplo es de orden $7$ por $7$, siendo 

$$  L = (\alpha dt L +I )$$

siendo en el código: $ L = alpha*dt*L + speye(size(L))$; finalmente  para obtener la solución 

 \begin{equation}	
	u^{n+1}_{i} = L *u^{n}_{i}
\end{equation}
 
 en el código se tiene $ U = L*U$






\section{Convection-diffusion}

\chapter{Tests}

\chapter{Python}

\chapter{Fortran}

\chapter{C++}

\chapter{Octave}

\nocite{*}
\printbibliography[title={Referencias}]

\end{document}
