%! arara: lualatex: {
%! arara: --> shell: yes,
%! arara: --> draft: yes,
%! arara: --> interaction: batchmode
%! arara: --> }
\documentclass[a4paper,abstract=true]{scrreprt}
\usepackage{fontspec}
\usepackage[svgnames]{xcolor}
\usepackage{graphicx}
\usepackage{diffcoeff}
\usepackage{minted}
\usepackage{hyperref}

\title{\color{DarkBlue}MOLE Tutorial}
\titlehead{\centering    
\includegraphics[width=.3\paperwidth]{logo.png}}
\author{based on commit \href{https://github.com/csrc-sdsu/mole/tree/b1176d9969e807fb62bed8fee28bc0eb9648a93a}{\mintinline{bash}|b1176d9|}}

\setminted{breaklines=true}
\setminted[octave]{highlightcolor=yellow!50!white}
\setminted[cpp]{highlightcolor=yellow!50!white}
\setmonofont{Fira Code}[Contextuals=Alternate,Scale=MatchLowercase]

\begin{document}
\maketitle

\begin{abstract}
    This document is written for a novice users in
    \href{https://octave.org}{GNU Octave}/\href{https://www.mathworks.com/products/matlab.html}{MATLAB}
    and C++ but with a solid
    foundation in numerical analysis for PDEs.
    MOLE stands Mimetic Operators Library Enhanced.
    % for gave.
\end{abstract}

\part{First part}

\url{https://csrc-sdsu.github.io/mole}

\part{Numerical Exercises}

\section{Convection-diffusion}

\chapter{Tests}

\chapter{Python}

\chapter{Fortran}

\chapter{C++}

\chapter{Octave}

\end{document}