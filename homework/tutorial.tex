% arara: clean: {
% arara: --> extensions:
% arara: --> ['aux', 'bbl', 'bcf', 'blg','idx', 'log', 'mst', 'pdf', 'run.xml']
% arara: --> }
%! arara: lualatex: {
%! arara: --> shell: yes,
%! arara: --> draft: yes,
%! arara: --> interaction: batchmode
%! arara: --> }
%! arara: biber
% arara: lualatex: {
% arara: --> shell: yes,
% arara: --> draft: no,
% arara: --> interaction: batchmode
% arara: --> }
%! arara: lualatex: {
%! arara: --> shell: yes,
%! arara: --> draft: no,
%! arara: --> interaction: batchmode
%! arara: --> }
% arara: clean: {
% arara: --> extensions:
% arara: --> ['aux', 'bbl', 'bcf', 'blg','idx', 'log', 'mst', 'run.xml']
% arara: --> }
\documentclass[a4paper,abstract=true]{scrreprt}
\usepackage{fontspec}
\usepackage[svgnames]{xcolor}
\usepackage{graphicx}
\usepackage{diffcoeff}
\usepackage{minted}
\usepackage{hyperref}

\title{\color{DarkBlue}MOLE Tutorial}
\titlehead{\centering    
\includegraphics[width=.3\paperwidth]{logo.png}}
\author{based on commit \href{https://github.com/csrc-sdsu/mole/tree/b1176d9969e807fb62bed8fee28bc0eb9648a93a}{\mintinline{bash}|b1176d9|}}

\setminted{breaklines=true}
\setminted[octave]{highlightcolor=yellow!50!white}
\setminted[cpp]{highlightcolor=yellow!50!white}
\setmonofont{Fira Code}[Contextuals=Alternate,Scale=MatchLowercase]

\begin{document}
\maketitle

\begin{abstract}
    This document is written for newcomers in MOLE stands
    Mimetic Operators Library Enhanced, but with a solid foundation
    in numerical analysis for PDEs.
    The algorithms are continuously grows in size and quality and the
    examples are diverse implemented independently in
    \href{https://octave.org}{GNU Octave}/\href{https://www.mathworks.com/products/matlab.html}{MATLAB}
    and \href{https://isocpp.org}{C++} through Armadillo sparse
    linear algebra library.
    For now, these are exclusively the two language flavours.
    A clear comprenhension will gave you the capabilities to the translation
    into other high-performance scientific languages such as \href{https://julialang.org}{Julia},
    \href{https://www.python.org}{Python}, \href{https://fortran-lang.org}{Fortran}, \href{https://www.open-std.org/jtc1/sc22/wg14}{C}
    and \href{https://www.rust-lang.org}{Rust}.
\end{abstract}

\part{First part}

\section{Meeting MOLE}

The official website is \url{https://csrc-sdsu.github.io/mole}.
After skimming the description and reading the papers you will find out that this method never uses a ghost points.

\subsection{Create the staggered grid}


\part{Numerical Exercises}

\section{Ecuación de diffusion 1D}
Se tiene  la ecuación de calor unidimensional
\begin{center}
$u_{t} = k  u_{xx} $, $0 < x < \pi$, $0 < t< 1$	
\end{center}

C.I : $u(x,0) = 100$\\

C.C:  $u(0,t)=0 =u(\pi,t)$\\

donde $ k =1$ es la difusividad térmica.
Usando métodos mimeticos 


\section{Convection-diffusion}

\chapter{Tests}

\chapter{Python}

\chapter{Fortran}

\chapter{C++}

\chapter{Octave}

\end{document}
