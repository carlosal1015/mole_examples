\chapter{MOLE Documentation}

\section{Mimetic Operator's Library Enhanced Reference}

We split the MOLE documentation in three categories:

\begin{description}
	\item[\href{https://carlosal1015.github.io/mole_examples/html}{General MOLE documentation}]

	      It contains general information and examples.

	\item[\href{https://carlosal1015.github.io/mole_examples/doxygen/cpp/html}{C++ MOLE documentation}]

	      It contains API C++ Reference.

	\item[\href{https://carlosal1015.github.io/mole_examples/doxygen/matlab}{Octave MOLE documentation}]

	      It contains API GNU/Octave Reference.
\end{description}

\section{Programming languages documentation}

\begin{description}
	\item[\href{https://en.cppreference.com}{C++ docs}]

	      \mintinline{cpp}|#include <iostream>|

	      \mintinline{cpp}|#include <cmath>|

	      \mintinline{cpp}|#include <vector>|.

	\item[\href{https://docs.octave.org}{Octave docs}]

	      \mintinline{octave}|addpath("/usr/share/")|.

	\item[\href{https://www.mathworks.com/help/matlab/index.html}{MATLAB docs}]

	      .
\end{description}

\section{Linear Algebra software documentation}

\begin{description}
	\item[\href{https://www.intel.com/content/www/us/en/developer/tools/oneapi/onemkl-documentation.html}{Intel MKL docs}]

	      .

	\item[\href{http://www.openmathlib.org/OpenBLAS/docs}{Openblas docs}]

	      .

	\item[\href{https://www.netlib.org/lapack/explore-html}{Netlib Lapack docs}]

	      .

	\item[\href{https://portal.nersc.gov/project/sparse/superlu/superlu_code_html/index.html}{SuperLU docs}]

	      .

	\item[\href{https://eigen.tuxfamily.org/dox}{Eigen docs}]

	      .

	\item[\href{https://arma.sourceforge.net/docs.html}{Armadillo docs}]

	      .
\end{description}

\begin{description}
	\item[\href{https://google.github.io/googletest}{Gtest docs}]

	      .

	\item[\href{https://docs.scipy.org/doc/scipy/reference/sparse.html}{SciPy sparse docs}]

	      .

	\item[\href{https://matplotlib.org/stable/api/_as_gen/matplotlib.pyplot.plot.html}{Matplotlib docs}]

	      .

	\item[\href{https://docs.h5py.org/en/stable}{HDF5 for Python docs}]

	      .
\end{description}

\chapter{Sparse Linear Algebra software examples}

\section{Armadillo}

We follow this gentle
\href{https://anderkve.github.io/FYS3150/book/introduction_to_cpp/intro_to_armadillo.html}{\emph{Introduction to Armadillo}}.

\subsection{Vectors}

\begin{listing}[ht!]
	\tiny
	\centering
	\pathinputminted[frame=single,framesep=12pt,linenos,firstline=1,lastline=56,highlightlines={43-44}]{cpp}{1.cc}
	\caption{Program~\texttt{1.cc}}
	\label{code:1.m}
\end{listing}

\subsection{Matrices}

\subsubsection{Dense Matrices}

\subsubsection{Sparse Matrices}

\subsubsection{Solvers}

\section{Eigen}

\section{SciPy Sparse}

See \href{https://github.com/nutrik/pymole}{pymole}.

\section{Sparse Arrays Julia}

See \href{https://robertsweeneyblanco.github.io/Programming_for_Mathematical_Applications/content/Sparse_Matrices/Sparse_Matrices_In_Julia.html}{}

\section{Fortran Sparse}

% https://www.intel.com/content/www/us/en/docs/onemkl/developer-reference-fortran/2024-0/inspector-executor-sparse-blas-routines.html
\section{Sparse Linear Algebra in Rust}

%https://github.com/sparsemat/sprs
%https://www.nalgebra.org

\section{PETSc sparse matrices in C}
