\chapter{Core Mimetic Operators}

\section{Divergence operator}

\subsection{1D Formulation}
The one-dimensional mimetic divergence operator $div(k,m,k)$, where $k$: order of accuracy, $m$: number of cells and $dx$: step size. The operator $div$ is a matrix of order $m+2$ by $m+1$. Where $k$ can take values $​​2, 4, 6$ and $8$; $m > 2*k+1$, that is:\\

if $k=2$, then the minimum value it takes $m=5$\\
if $k=4$, then the minimum value it takes $m=9$\\
if $k=6$, then the minimum value it takes $m=13$\\
if $k=8$, then the minimum value it takes $m=17$\\
\begin{listing}[ht!]
	\tiny
	\centering
	\pathinputminted[frame=single,framesep=10pt,linenos]{octave}{div.m}
	\caption{Program~\texttt{div.m}}
	\label{code:div.m}
\end{listing}


Example: If  $K=2$, $m=5$ and $dx=1$, then the divergence matrix is ​​of order 7 by 6.

\begin{equation}
D= (\frac{1}{dx})\begin{pmatrix}
0 & 0 & 0 & 0 & 0 & 0\\
-1 & 1 & 0 & 0 & 0 & 0\\
0 & -1 & 1 & 0 & 0 & 0\\
0 & 0 & -1 & 1 & 0 & 0\\
0 & 0 & 0 & -1 & 1 & 0\\
0 & 0 & 0 & 0 & -1 & 1\\
0 & 0 & 0 & 0 & 0 & 0\\
\end{pmatrix}
\end{equation}



\begin{listing}[ht!]
	\tiny
	\centering
	\pathinputminted[frame=single,framesep=10pt,linenos]{octave}{divergence.h}
	\caption{Program~\texttt{divergence.h}}
	\label{code:divergence.h}
\end{listing}

\begin{listing}[ht!]
	\tiny
	\centering
	\pathinputminted[frame=single,framesep=10pt,linenos,firstline=1,lastline=100]{python}{div1D.py}
	\caption{Program~\texttt{div1D.py}}
	\label{code:div1D.py}
\end{listing}

\subsection{2D Formulation}

\begin{listing}[ht!]
	\tiny
	\centering
	\pathinputminted[frame=single,framesep=10pt,linenos]{octave}{div2D.m}
	\caption{Program~\texttt{div2D.m}}
	\label{code:div2D.m}
\end{listing}

\subsection{3D Formulation}

\begin{listing}[ht!]
	\tiny
	\centering
	\pathinputminted[frame=single,framesep=10pt,linenos]{octave}{div3D.m}
	\caption{Program~\texttt{div3D.m}}
	\label{code:div3D.m}
\end{listing}

\section{Gradient Operator}

\subsection{1D Formulation}

\begin{listing}[ht!]
	\tiny
	\centering
	\pathinputminted[frame=single,framesep=10pt,linenos]{octave}{grad.m}
	\caption{Program~\texttt{grad.m}}
	\label{code:grad.m}
\end{listing}

\begin{listing}[ht!]
	\tiny
	\centering
	\pathinputminted[frame=single,framesep=10pt,linenos]{octave}{gradient.h}
	\caption{Program~\texttt{gradient.h}}
	\label{code:gradient.h}
\end{listing}

\begin{listing}[ht!]
	\tiny
	\centering
	\pathinputminted[frame=single,framesep=10pt,linenos,firstline=1,lastline=100]{python}{grad1D.py}
	\caption{Program~\texttt{grad1D.py}}
	\label{code:grad1D.py}
\end{listing}

\subsection{2D Formulation}

\begin{listing}[ht!]
	\tiny
	\centering
	\pathinputminted[frame=single,framesep=10pt,linenos]{octave}{grad2D.m}
	\caption{Program~\texttt{grad2D.m}}
	\label{code:grad2D.m}
\end{listing}

\subsection{3D Formulation}

\begin{listing}[ht!]
	\tiny
	\centering
	\pathinputminted[frame=single,framesep=10pt,linenos]{octave}{grad3D.m}
	\caption{Program~\texttt{grad3D.m}}
	\label{code:grad3D.m}
\end{listing}

\section{Laplacian Operator}

\subsection{1D Formulation}

\begin{listing}[ht!]
	\tiny
	\centering
	\pathinputminted[frame=single,framesep=10pt,linenos]{octave}{lap.m}
	\caption{Program~\texttt{lap.m}}
	\label{code:lap.m}
\end{listing}

\begin{listing}[ht!]
	\tiny
	\centering
	\pathinputminted[frame=single,framesep=10pt,linenos]{octave}{laplacian.h}
	\caption{Program~\texttt{laplacian.h}}
	\label{code:laplacian.h}
\end{listing}

\begin{listing}[ht!]
	\tiny
	\centering
	\pathinputminted[frame=single,framesep=10pt,linenos,firstline=1,lastline=18]{python}{lap1D.py}
	\caption{Program~\texttt{lap1D.py}}
	\label{code:lap1D.py}
\end{listing}

\subsection{2D Formulation}

\begin{listing}[ht!]
	\tiny
	\centering
	\pathinputminted[frame=single,framesep=10pt,linenos]{octave}{lap2D.m}
	\caption{Program~\texttt{lap2D.m}}
	\label{code:lap2D.m}
\end{listing}

\subsection{3D Formulation}

\begin{listing}[ht!]
	\tiny
	\centering
	\pathinputminted[frame=single,framesep=10pt,linenos]{octave}{lap3D.m}
	\caption{Program~\texttt{lap3D.m}}
	\label{code:lap3D.m}
\end{listing}

\section{Interpolation Operators}

\begin{listing}[ht!]
	\tiny
	\centering
	\pathinputminted[frame=single,framesep=10pt,linenos]{octave}{interpol.m}
	\caption{Program~\texttt{interpol.m}}
	\label{code:interpol.m}
\end{listing}

\begin{listing}[ht!]
	\tiny
	\centering
	\pathinputminted[frame=single,framesep=10pt,linenos]{octave}{interpol2D.m}
	\caption{Program~\texttt{interpol2D.m}}
	\label{code:interpol2D.m}
\end{listing}

\begin{listing}[ht!]
	\tiny
	\centering
	\pathinputminted[frame=single,framesep=10pt,linenos]{octave}{interpol3D.m}
	\caption{Program~\texttt{interpol3D.m}}
	\label{code:interpol3D.m}
\end{listing}

\subsection{Center $\Longleftrightarrow$ Nodes (1D, 2D, 3D)}
\subsection{Center $\Longleftrightarrow$ Faces (1D, 2D, 3D)}
\subsection{Nodes $\Longleftrightarrow$ Faces (1D, 2D, 3D)}

% interpolCentersToFacesD1D.m
% \section{Interpolation 1D from center to nodes}
% \section{Interpolation 2D from center to nodes}
% \section{Interpolation 3D from center to nodes}
% \section{Interpolation 1D from center to faces}
% \section{Interpolation 2D from center to faces}
% \section{Interpolation 3D from center to faces}
% \section{Interpolation 1D from nodes to center}
% \section{Interpolation 2D from nodes to center}
% \section{Interpolation 3D from nodes to center}
% \section{Interpolation 1D from faces to center}
% \section{Interpolation 2D from faces to center}
% \section{Interpolation 3D from faces to center}

\section{Boundary Condition Implementation}

\subsection{1D Boundary Handling}

\begin{listing}[ht!]
	\tiny
	\centering
	\pathinputminted[frame=single,framesep=10pt,linenos]{octave}{addBC1D.m}
	\caption{Program~\texttt{addBC1D.m}}
	\label{code:addBC1D.m}
\end{listing}

\subsection{2D/3D Boundary Handling}
