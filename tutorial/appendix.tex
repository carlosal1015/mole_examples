\appendix

\chapter{Installing MOLE on Linux}

To work through this tutorial requires to have a working installation
of the MOLE software.
It relies on Armadillo~\cite{Sanderson2025}, a fast C++ library for
numerical linear algebra. % https://gitlab.com/conradsnicta/armadillo-code
We believe that the best choice to getting started in scientific
computing is GNU/Linux.
There are \href{https://upload.wikimedia.org/wikipedia/commons/1/1b/Linux_Distribution_Timeline.svg}{several flavours},
but we show the steps process for Ubuntu
Linux~\footnote{See~\url{https://help.ubuntu.com/lts/ubuntu-help/index.html}.}
and Arch Linux~\footnote{See~\url{https://wiki.archlinux.org/title/Pacman/Rosetta}.}
since both systems were sucessfully tested.
Let's show the steps in order to get a proper installation of MOLE.
% \href{https://books.goalkicker.com/LinuxBook}{Linux commands Notes for Professionals book}.

\section{Ubuntu Linux}

The user need

\begin{listing}[ht!]
	\tiny
	\centering
	\pathinputminted[frame=single,framesep=10pt,linenos,firstline=1,lastline=18,highlightlines={9,15}]{bash}{installer.sh}
	\caption{Steps for a system-wide installation both C++ and Octave
		MOLE libraries vía \href{https://raw.githubusercontent.com/carlosal1015/mole_examples/main/homework/installer.sh}{\texttt{installer.sh}} on
		\href{https://archlinux.org}{Arch Linux}.}
	\label{code:installer.sh}
\end{listing}

\section{Arch Linux}

The only hard dependency is
\href{https://arma.sourceforge.net/docs.html}{\mintinline{cpp}|armadillo|}.
The steps are described in the Program~\ref{code:installer.sh}.

\begin{listing}[ht!]
	\tiny
	\centering
	\pathinputminted[frame=single,framesep=10pt,linenos,firstline=1,lastline=18,highlightlines={9,15}]{bash}{installer.sh}
	\caption{Steps for a system-wide installation both C++ and Octave
		MOLE libraries vía \href{https://raw.githubusercontent.com/carlosal1015/mole_examples/main/homework/installer.sh}{\texttt{installer.sh}} on
		\href{https://archlinux.org}{Arch Linux}.}
	\label{code:installer.sh}
\end{listing}

\section{Docker image ideal for Windows 11}

%docker pull ghcr.io/carlosal1015/mole_examples/libmole-git

\section{MOLE Documentation}

\section{Getting started with MOLE}

\subsection{Compiling and running the first code}

\chapter{Julia}

\chapter{Python}

\chapter{Fortran}

\nocite{*}
\printbibliography[title={References}]
