\appendix

\chapter{Installing MOLE software on GNU/Linux}

To work through this tutorial requires to have a working installation
of MOLE.
It relies on
\href{https://gitlab.com/conradsnicta/armadillo-code}{\mintinline{cpp}|armadillo|}~\cite{Sanderson2025},
a C++ library that provides data structures for sparse matrices.
We explain the step-by-step process for
\href{https://wiki.archlinux.org/title/Pacman/Rosetta}{Arch Linux}
and \href{https://help.ubuntu.com/lts/ubuntu-help/index.html}{Ubuntu Linux},
as both systems have been sucessfully tested by us.
% We believe that the best choice for getting started in scientific computing is GNU/Linux.
% \href{https://books.goalkicker.com/LinuxBook}{Linux commands Notes for Professionals book}.

\section{Arch Linux}

This distribution is supported by a proactive group of
\href{https://archlinux.org/people/developers}{developers},
\href{https://archlinux.org/people/package-maintainers}{package maintainers}
and \href{https://archlinux.org/people/support-staff}{support staff}
that try to provides the latest stable software releases.
The steps are outlined in the Program~\ref{code:installerarchlinux.sh}.

\begin{listing}[ht!]
	\tiny
	\centering
	\pathinputminted[frame=single,framesep=10pt,linenos,firstline=1,lastline=18,highlightlines={9,15},escapeinside=||]{bash}{installerarchlinux.sh}
	\caption{Steps for a system-wide installation both C++ and Octave
		MOLE library vía
		\href{https://raw.githubusercontent.com/carlosal1015/mole_examples/main/tutorial/installerarchlinux.sh}{\texttt{installerarchlinux.sh}}.}
	\label{code:installerarchlinux.sh}
\end{listing}

Even if you are using Windows, the
\href{https://docs.docker.com/desktop/features/wsl}{Docker Desktop WSL 2 backend}
is ideal for using MOLE via Program~\ref{code:docker.sh} or
\href{https://wiki.archlinux.org/title/Install_Arch_Linux_on_WSL}{installing Arch Linux on WSL 2} and following
the Program~\ref{code:installerarchlinux.sh}.

\begin{listing}[ht!]
	\tiny
	\centering
	\pathinputminted[frame=single,framesep=10pt,linenos,firstline=1,lastline=7,highlightlines={3}]{bash}{docker.sh}
	\caption{Pull container based on Arch Linux with set up MOLE
		library vía \href{https://raw.githubusercontent.com/carlosal1015/mole_examples/main/tutorial/docker.sh}{\texttt{docker.sh}}.}
	\label{code:docker.sh}
\end{listing}

\section{Ubuntu Linux}

This \href{https://www.debian.org}{Debian}-derived distribution is
managed by \href{https://canonical.com}{Canonical Ltd.}
Each 2 years they launch a Long Term Support(LTS) release.
The steps are outlined in the Program~\ref{code:installerubuntu.sh}.

\begin{listing}[ht!]
	\tiny
	\centering
	\pathinputminted[frame=single,framesep=10pt,linenos,firstline=1,lastline=25,highlightlines={9,15}]{bash}{installerubuntu.sh}
	\caption{Steps for a system-wide installation both C++ and Octave
		MOLE library vía \href{https://raw.githubusercontent.com/carlosal1015/mole_examples/main/tutorial/installerubuntu.sh}{\texttt{installerubuntu.sh}}.}
	\label{code:installerubuntu.sh}
\end{listing}

\chapter{Documentation}

We split the documentation in three categories.

\section{MOLE documentation}

\begin{description}
	\item[General docs]

	      \url{https://carlosal1015.github.io/mole_examples/html}

	\item[C++ docs]

	      \url{https://carlosal1015.github.io/mole_examples/doxygen/cpp/html}

	\item[Octave docs]

	      \url{https://carlosal1015.github.io/mole_examples/doxygen/matlab}
\end{description}

\section{Programming language documentation}

\begin{itemize}
	\item \href{https://en.cppreference.com}{C++ docs}

	      \mintinline{cpp}|#include <iostream>|

	      \mintinline{cpp}|#include <cmath>|

	      \mintinline{cpp}|#include <vector>|

	\item \href{https://docs.octave.org}{Octave docs}

	      \mintinline{octave}|addpath("/usr/share/")|

	\item \href{https://www.mathworks.com/help/matlab/index.html}{MATLAB docs}
\end{itemize}

\section{Linear Algebra software documentation}

\begin{description}
	\item[Intel MKL docs]

	      \url{https://www.intel.com/content/www/us/en/developer/tools/oneapi/onemkl-documentation.html}

	\item[Openblas docs]

	      \url{http://www.openmathlib.org/OpenBLAS/docs}

	\item[Netlib Lapack docs]

	      \url{https://www.netlib.org/lapack/explore-html}

	\item[SuperLU docs]

	      \url{https://portal.nersc.gov/project/sparse/superlu/superlu_code_html/index.html}

	\item[Eigen docs]

	      \url{https://eigen.tuxfamily.org/dox}

	\item[Armadillo docs]
	      \url{https://arma.sourceforge.net/docs.html}
\end{description}

\section{Extra documentation}

\begin{description}
	\item[Gtest docs]

	      \url{https://google.github.io/googletest}

	\item[SciPy sparse docs]

	      \url{https://docs.scipy.org/doc/scipy/reference/sparse.html}

	\item[Matplotlib docs]

	      \url{https://matplotlib.org/stable/api/_as_gen/matplotlib.pyplot.plot.html}

	\item[H5DF for Python docs]
	      \url{https://docs.h5py.org/en/stable}
\end{description}

\chapter{Armadillo Crash Course}

We follow this gentle
\href{https://anderkve.github.io/FYS3150/book/introduction_to_cpp/intro_to_armadillo.html}{\emph{Introduction to Armadillo}}.

\section{Vectors}

\section{Matrices}

\subsection{Dense Matrices}

\subsection{Sparse Matrices}

\section{Solvers}

\chapter{Julia bindings}

\chapter{Python bindings}

See \href{% https://github.com/nutrik/pymole}{pymole}.

\chapter{Fortran bindings}

\nocite{*}
\printbibliography[title={References}]
