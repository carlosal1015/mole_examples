\part{Validation and Benchmarking}

% TODO: Comparar algunas soluciones con el método de diferencias finitas, como por ejemplo, para la ecuación de onda.
% TODO: Ejemplo, donde el método numérico es conservativo (se conserva la energía)
\chapter{Operator Validation}

\section{Divergence Operator Tests}

\subsection{Accuracy in 1D}

\subsection{Conservation in 2D}

\subsection{Boundary Compliance in 3D}

\section{Gradient and Laplacian Tests}

\section{Interpolation Robustness Checks}

\chapter{Application-Driven Tests}

\section{Stability Analysis for PDE Solvers}

% \section{Performance Metrics}
% \subsection{Computational Efficiency}
% \subsection{Scalability in High Dimensions}

MOLE uses \href{https://google.github.io/googletest}{GoogleTest} framework.

\begin{listing}[ht!]
	\tiny
	\centering
	\pathinputminted[frame=single,framesep=10pt,linenos,firstline=1,lastline=33,highlightlines={9-19}]{cpp}{test1.cpp}
	\caption{Program~\texttt{test1.cpp}}
	\label{code:test1.cpp}
\end{listing}

\begin{listing}[ht!]
	\tiny
	\centering
	\pathinputminted[frame=single,framesep=10pt,linenos,firstline=1,lastline=21,highlightlines={8-19}]{octave}{test1.m}
	\caption{Program~\texttt{test1.m}}
	\label{code:test1.m}
\end{listing}

\begin{listing}[ht!]
	\tiny
	\centering
	\pathinputminted[frame=single,framesep=10pt,linenos,firstline=1,lastline=28,highlightlines={9-20}]{cpp}{test2.cpp}
	\caption{Program~\texttt{test2.cpp}}
	\label{code:test2.cpp}
\end{listing}

\begin{listing}[ht!]
	\tiny
	\centering
	\pathinputminted[frame=single,framesep=10pt,linenos,firstline=1,lastline=21,highlightlines={8-19}]{octave}{test2.m}
	\caption{Program~\texttt{test2.m}}
	\label{code:test2.m}
\end{listing}

\begin{listing}[ht!]
	\tiny
	\centering
	\pathinputminted[frame=single,framesep=10pt,linenos,firstline=1,lastline=27,highlightlines={8-19}]{cpp}{test3.cpp}
	\caption{Program~\texttt{test3.cpp}}
	\label{code:test3.cpp}
\end{listing}

\begin{listing}[ht!]
	\tiny
	\centering
	\pathinputminted[frame=single,framesep=10pt,linenos,firstline=1,lastline=21,highlightlines={8-19}]{octave}{test3.m}
	\caption{Program~\texttt{test3.m}}
	\label{code:test3.m}
\end{listing}

\begin{listing}[ht!]
	\tiny
	\centering
	\pathinputminted[frame=single,framesep=10pt,linenos,firstline=1,lastline=39,highlightlines={10-39}]{cpp}{test4.cpp}
	\caption{Program~\texttt{test4.cpp}}
	\label{code:test4.cpp}
\end{listing}

\begin{listing}[ht!]
	\tiny
	\centering
	\pathinputminted[frame=single,framesep=10pt,linenos,firstline=1,lastline=48,highlightlines={15-44}]{octave}{test4.m}
	\caption{Program~\texttt{test4.m}}
	\label{code:test4.m}
\end{listing}

\begin{listing}[ht!]
	\tiny
	\centering
	\pathinputminted[frame=single,framesep=10pt,linenos,firstline=1,lastline=61,highlightlines={9-53}]{cpp}{test5.cpp}
	\caption{Program~\texttt{test5.cpp}}
	\label{code:test5.cpp}
\end{listing}

\begin{listing}[ht!]
	\tiny
	\centering
	\pathinputminted[frame=single,framesep=10pt,linenos,firstline=1,lastline=55,highlightlines={10-53}]{octave}{test5.m}
	\caption{Program~\texttt{test5.m}}
	\label{code:test5.m}
\end{listing}

