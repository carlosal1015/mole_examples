\documentclass[a4paper,abstract=true]{scrreprt}

\usepackage{fontspec}
\usepackage[svgnames]{xcolor}
\usepackage[many]{tcolorbox}
\usepackage{graphicx}
\usepackage{academicons}
\usepackage{background}
\usepackage{todonotes}
\usepackage[shortlabels]{enumitem}
\usepackage{booktabs}
\usepackage[ISO]{diffcoeff}
\usepackage{mathtools}
\usepackage{amssymb}
\usepackage{bm}
\usepackage{dsfont}
\usepackage[makeroom]{cancel}
\usepackage{nicematrix}
\usepackage{amsthm}
\usepackage[linesnumbered,ruled,boxed,vlined]{algorithm2e}
\usepackage{algorithmicx}
\usepackage{minted}
\usepackage[
	citestyle=numeric,
	style=numeric,
	backend=biber,
]{biblatex}
\usepackage{hyperref}
\usepackage{cleveref}

\title{
	\color{DarkBlue}Tutorial for MOLE
}
\subtitle{
	\color{NavyBlue}
	based on commit \href{https://github.com/csrc-sdsu/mole/tree/723d18bac90696260bd29b12d04f03605a997557}{\mintinline{bash}|723d18b|}
}
\titlehead{\centering
	\href{https://csrc-sdsu.github.io/mole}{\includegraphics[width=.3\paperwidth]{logo.png}}}

\author{
	Carlos Aznarán\thanks{
		Universidad Nacional de Ingeniería \\
		Av. Tupac Amaru, s/n, Lima, Perú \\
		\href{mailto:caznaranl@uni.pe}{\texttt{caznaranl\MVAt uni.pe}} \\
		\textcolor{orcidcolor}{\aiOrcid}~\href{https://orcid.org/0000-0001-8314-2271}{0000-0001-8314-2271}}
	\and Adelaida Otazu\thanks{
		Universidad Nacional del Altiplano Puno \\
		Avenida El Sol, Puno, Perú \\
		\href{mailto:aotazu@unap.edu.pe}{\texttt{aotazu\MVAt unap.edu.pe}} \\
		\textcolor{orcidcolor}{\aiOrcid}~\href{https://orcid.org/0000-0003-4793-0400}{0000-0003-4793-0400}}
}

\setminted{breaklines=true}
\setminted[octave]{highlightcolor=yellow!50!white}
\setminted[cpp]{highlightcolor=yellow!50!white}
\setmonofont{Fira Code}[Contextuals=Alternate,Scale=MatchLowercase]
\addbibresource{references.bib}

\renewcommand{\listingscaption}{Program}

\makeatletter
% latex.ltx, line 8262:
\def\@fnsymbol#1{%
	\ifcase#1\or
		%\TextOrMath\textasteriskcentered *\or
		\TextOrMath \textdagger \dagger\or
		\TextOrMath \textdaggerdbl \ddagger \or
		\TextOrMath \textsection  \mathsection\or
		\TextOrMath \textparagraph \mathparagraph\or
		\TextOrMath \textbardbl \|\or
		%\TextOrMath {\textasteriskcentered\textasteriskcentered}{**}\or
		\TextOrMath {\textdagger\textdagger}{\dagger\dagger}\or
		\TextOrMath {\textdaggerdbl\textdaggerdbl}{\ddagger\ddagger}\else
		\@ctrerr \fi
}
\makeatother

\newcommand{\MVAt}{{\usefont{U}{mvs}{m}{n}\symbol{`@}}}
\definecolor{orcidcolor}{HTML}{a5ca3f}
\definecolor{orcidlinkcolor}{HTML}{0961ba}

\difdef{c}{L}{op-symbol=\mathop{}\!\mathbin\bigtriangleup}
\difdef{c}{A}{op-symbol=\mathop{}\!\mathbin\Box}

\ExplSyntaxOn
\NewDocumentCommand{\mintedpath}{m}
{
	\seq_gset_split:Nnn \g_paulie_mintedpath_seq { } { #1 }
	\seq_gput_left:Nn \g_paulie_mintedpath_seq { }
}

\seq_new:N \g_paulie_mintedpath_seq

\NewDocumentCommand{\pathinputminted}{O{}mm}
{
	\seq_map_inline:Nn \g_paulie_mintedpath_seq
	{
		\file_if_exist:nT { ##1 #3 }
		{
			\inputminted[#1]{#2}{##1 #3}
			\seq_map_break:
		}
	}
}
\ExplSyntaxOff

\mintedpath{{../examples/cpp/}{../examples/octave/}{../homework/}{../tests/cpp/}{../tests/octave/}}
\graphicspath{{../homework/}{../examples/octave/}}

\backgroundsetup{
	scale=1,
	angle=0,
	opacity=.1,
	contents={
			\ifodd\value{page}
				\begin{tikzpicture}[remember picture,overlay]
					\node at ([yshift=11pt,xshift=5pt]current page.center)
					{\includegraphics[width=8.1cm]{logouni}};
				\end{tikzpicture}
			\else
				\begin{tikzpicture}[remember picture,overlay]
					\node at ([yshift=11pt,xshift=5pt]current page.center)
					{\includegraphics[width=8.1cm]{logounap}};
				\end{tikzpicture}
			\fi
		}
}

\tcbset{
	defstyle/.style={
			fonttitle=\bfseries\upshape,
			arc=0mm,
			colback=blue!5!white,
			colframe=blue!75!black
		},
	theostyle/.style={
			fonttitle=\bfseries\upshape,
			arc=0mm,%fontupper=\slshape,
			colback=red!10!white,
			colframe=red!75!black
		}
}
\newtcbtheorem[
	crefname={definición}{definitions}
]
{definition}{Definición}{defstyle}{def}

\newtcbtheorem[
	crefname={teorema}{theorems}
]
{theorem}{Teorema}{theostyle}{thm}

\newtcbtheorem[
	crefname={ejemplo}{examples}
]{example}{Ejemplo}{
	enhanced,
	frame empty,
	interior empty,
	colframe=ForestGreen!50!white,
	coltitle=ForestGreen!50!black,
	fonttitle=\bfseries,colbacktitle=ForestGreen!15!white,
	borderline={0.5mm}{0mm}{ForestGreen!15!white},
	borderline={0.5mm}{0mm}{ForestGreen!50!white,dashed},
	attach boxed title to top center={yshift=-2mm},
	boxed title style={boxrule=0.4pt},
	varwidth boxed title}{theo}

\theoremstyle{definition}
\newtheorem{corollary}{Corolario}
\newtheorem{proposition}{Proposición}
\newtheorem{axiom}{Axioma}
\newtheorem{remark}{Observación}
\newtheorem{note}{Nota}
\newtheorem{property}{Propiedad}
\newtheorem{lemma}{Lema}

\tcolorboxenvironment{corollary}{
	enhanced jigsaw,
	colframe=cyan,
	interior hidden,
	breakable,
	before skip=10pt,
	after skip=10pt
}

\tcolorboxenvironment{proof}{
	blanker,
	breakable,
	left=5mm,
	before skip=10pt,
	after skip=10pt,
	borderline west={1mm}{0pt}{red}
}

\renewcommand{\qedsymbol}{\(\blacksquare\)}
